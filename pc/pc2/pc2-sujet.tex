%-*- coding: utf-8 -*-
\documentclass[french,11pt]{article}
\usepackage{babel}
\DecimalMathComma

\usepackage{graphicx}
\usepackage[hidelinks]{hyperref}


% Fonts
\usepackage[T1]{fontenc}
\usepackage{gentium-otf}

% SI units
\usepackage{siunitx}

% Table becomes Tableau
\usepackage{caption}
\captionsetup{labelfont=sc}
\def\frenchtablename{Tableau}

% % List management
\usepackage{enumitem}

\usepackage[dvipsnames]{xcolor}
\usepackage{listings}
\lstset{%
  frame=single,                    % adds a frame around the code
  tabsize=2,                       % sets default tabsize to 2 spaces
  columns=flexible,                % doesn't add spaces to make the line fit the whole column
  basicstyle=\ttfamily,             % use monospace
  keywordstyle=\color{MidnightBlue},
  commentstyle=\color{Gray},
  stringstyle=\color{BurntOrange},
  showstringspaces=false,
}


%%%% GEOMETRY AND SPACING %%%%%%%%%%%%%%%%%%%%%%%%%%%%%%%%%%%%%%%%%%%%%%%
\usepackage{etex}
\usepackage[tmargin=2cm,bmargin=2cm,lmargin=2cm,footnotesep=1cm]{geometry}

\parskip=1ex\relax % space between paragraphs (incl. blank lines)
%%%%%%%%%%%%%%%%%%%%%%%%%%%%%%%%%%%%%%%%%%%%%%%%%%%%%%%%%%%%%%%%%%%%%%

%%% SYMBOLS %%%%%%%%%%%%%%%%%%%%%%%%%%%%%%%%%%%%%%%%%%%%%%%%%%%%%%%%%%
% % Math symbols
\usepackage{amsmath}
\usepackage{amssymb}
\usepackage{bm}

% Math symbols
\newcommand{\Acal}{\mathcal{A}}
\newcommand{\Bcal}{\mathcal{B}}
\newcommand{\Ccal}{\mathcal{C}}
\newcommand{\Ecal}{\mathcal{E}}
\newcommand{\Gcal}{\mathcal{G}}
\newcommand{\Ical}{\mathcal{I}}
\newcommand{\Kcal}{\mathcal{K}}
\newcommand{\Lcal}{\mathcal{L}}
\newcommand{\Ncal}{\mathcal{N}}
\newcommand{\Pcal}{\mathcal{P}}
\newcommand{\Qcal}{\mathcal{Q}}
\newcommand{\Rcal}{\mathcal{R}}
\newcommand{\Scal}{\mathcal{S}}
\newcommand{\Tcal}{\mathcal{T}}

\newcommand{\DD}{\mathcal{D}}
\newcommand{\FF}{\mathcal{F}}
\newcommand{\HH}{\mathcal{H}}
\newcommand{\LL}{\mathcal{L}}
\newcommand{\MM}{\mathcal{M}}
\newcommand{\OO}{\mathcal{O}}
\newcommand{\TT}{\mathcal{T}}
\newcommand{\UU}{\mathcal{U}}
\newcommand{\XX}{\mathcal{X}}
\newcommand{\YY}{\mathcal{Y}}
\newcommand{\ZZ}{\mathcal{Z}}

\newcommand{\CC}{\mathbb{C}}
\newcommand{\EE}{\mathbb{E}}
\newcommand{\NN}{\mathbb{N}}
\newcommand{\PP}{\mathbb{P}}
\newcommand{\RR}{\mathbb{R}}
\newcommand{\RRnp}{\mathbb{R}^{n \times p}}
\newcommand{\RRnn}{\mathbb{R}^{n \times n}}
\newcommand{\RRpp}{\mathbb{R}^{p \times p}}
\newcommand{\VV}{\mathbb{V}}


% Vectors
\makeatletter
\newcommand{\avec}{\vec{a}\@ifnextchar{^}{\,}{}}
\newcommand{\bb}{\vec{b}\@ifnextchar{^}{\,}{}}
\newcommand{\cvec}{\vec{c}\@ifnextchar{^}{\,}{}}
\newcommand{\gvec}{\vec{g}\@ifnextchar{^}{\,}{}}
\newcommand{\hh}{\vec{h}\@ifnextchar{^}{\,}{}}
\newcommand{\mm}{\vec{m}\@ifnextchar{^}{\,}{}}
\newcommand{\oo}{\vec{o}\@ifnextchar{^}{\,}{}}
\newcommand{\pp}{\vec{p}\@ifnextchar{^}{\,}{}}
\newcommand{\rr}{\vec{r}\@ifnextchar{^}{\,}{}}
\newcommand{\tvec}{\vec{t}\@ifnextchar{^}{\,}{}}
\newcommand{\uu}{\vec{u}\@ifnextchar{^}{\,}{}}
\newcommand{\uprime}{\vec{u'}\@ifnextchar{^}{\,}{}}
\newcommand{\vv}{\vec{v}\@ifnextchar{^}{\,}{}}
\newcommand{\vprime}{\vec{v'}\@ifnextchar{^}{\,}{}}
\newcommand{\ww}{\vec{w}\@ifnextchar{^}{\,}{}}
\newcommand{\xx}{\vec{x}\@ifnextchar{^}{\,}{}}
\newcommand{\yy}{\vec{y}\@ifnextchar{^}{\,}{}}
\newcommand{\zz}{\vec{z}\@ifnextchar{^}{\,}{}}
\newcommand{\aalpha}{\vec{\alpha}\@ifnextchar{^}{\,}{}}
\newcommand{\bbeta}{\vec{\beta}\@ifnextchar{^}{\,}{}}
\newcommand{\ttheta}{\vec{\theta}\@ifnextchar{^}{\,}{}}
\newcommand{\mmu}{\vec{\mu}\@ifnextchar{^}{\,}{}}
\newcommand{\xxi}{\vec{\xi}\@ifnextchar{^}{\,}{}}
\makeatother

% Hats
\newcommand{\thetahat}{{\widehat \theta}}
\newcommand{\hatn}{{\widehat n}}
\newcommand{\hatp}{{\widehat p}}
\newcommand{\hatm}{{\widehat m}}
\newcommand{\hatmu}{{\widehat \mu}}
\newcommand{\hatsigma}{{\widehat \sigma}}

\newcommand{\hatmle}[1]{\widehat{#1}_{\text{MLE}}}
\newcommand{\hatmap}[1]{\widehat{#1}_{\text{MAP}}}
\newcommand{\hatbys}[1]{\widehat{#1}_{\text{Bayes}}}

\DeclareMathOperator*{\argmax}{arg\,max}
\DeclareMathOperator*{\argmin}{arg\,min}

\newcommand{\cvn}{\xrightarrow[n \rightarrow +\infty]{}}
\newcommand{\cvproba}{\stackrel{\PP}{\longrightarrow}}
\newcommand{\cvps}{\stackrel{\text{p.s.}}{\longrightarrow}}
\newcommand{\cvltwo}{\stackrel{\Lcal^2}{\longrightarrow}}
\newcommand{\cvloi}{\stackrel{\Lcal}{\longrightarrow}}

% Sets
\newcommand{\zo}{\lbrace 0, 1 \rbrace}
\newcommand{\mopo}{\lbrace -1, 1 \rbrace}

% \newcommand{\lzeronorm}[1]{\left|\left|#1\right|\right|_0}
% \newcommand{\lonenorm}[1]{\left|\left|#1\right|\right|_1}
% \newcommand{\ltwonorm}[1]{\left|\left|#1\right|\right|_2}
% \newcommand{\lzeronorm}[1]{\|#1\|_0}
% \newcommand{\lonenorm}[1]{\|#1\|_1}
% \newcommand{\ltwonorm}[1]{\|#1\|_2}
\usepackage{mathtools}
\DeclarePairedDelimiter{\abs}{\lvert}{\rvert}
\DeclarePairedDelimiter{\norm}{\lVert}{\rVert}
\DeclarePairedDelimiter{\bignorm}{\bigg\lVert}{\bigg\rVert}
\DeclarePairedDelimiter{\innerproduct}{\langle}{\rangle}

\DeclarePairedDelimiter{\lzeronorm}{\lVert}{\rVert_0}
\DeclarePairedDelimiter{\lonenorm}{\lVert}{\rVert_1}
\DeclarePairedDelimiter{\ltwonorm}{\lVert}{\rVert_2}
%%%%%%%%%%%%%%%%%%%%%%%%%%%%%%%%%%%%%%%%%%%%%%%%%%%%%%%%%%%%%%%%%%%%%%


\begin{document}

\begin{center}
\bf\large ECUE21.2: Science des données \hfill
PC 2 -- Tests statistiques
\end{center}

\noindent
\hfill 13 juin 2025

\noindent
\rule{\textwidth}{.4pt}


\section*{Test du Chi2}
Un essai clinique sur 200 personnes, dont 92 ont été soumises au traitement
évalué, a mis en évidence que 84 d'entre elles n'ont plus de symptômes après
une semaine de traitement. 90 des personnes non traitées n'ont plus de
symptômes après une semaine non plus.

On cherche à déterminer si le traitement est efficace.

\begin{enumerate}
\item \textbf{Tables de contingence}
  \begin{enumerate}
  \item Établir la table de contingence observée correspondant à ces
    données. Quelle proportion de personnes traitées guérissent ? Quelle
    proportion de personnes non traitées guérissent ? Notre but sera de
    déterminer si cette différence est significative.
  \item Estimer la probabilité $p$ qu'une personne soit traitée. Estimer la
    probabilité $q$ qu'une personne guérisse (indépendamment du traitement).
  \item Supposer que le traitement n'a aucun effet. Quelle serait alors la
    table de contingence ?
  \item Interpréter la distance du chi2 de la table de contingence observée (cf
    section 2.2.1 du poly) comme une distance entre la table de contingence
    observée (a) et la table de contingence théorique (c).
  \end{enumerate}
\end{enumerate}

Soient $Y_1, Y_2, \dots, Y_k$ $k$ variables aléatoires réelles iid, suivant une
gaussienne standard. On pose
\[
  Z_k = \sum_{i=1}^k Y_i^2.
\]
On dit que $Z_k$ suit une loi du chi2 à $k$ degrés de liberté. On note
$Z_k \sim \chi_k^2$. (Cette loi vous a déjà été présentée dans les exercices de Probabilités II.)
Le tableau~\ref{tab:chi2} donne la valeur de $\PP(Z_k > z)$ pour quelques
valeurs de $k$ et de $z$.


On admettra\footnote{La question 3 de ce problème permet de démontrer
cette propriété dans le cas où on compare les proportions observées d'une
variable à deux modes aux proportions attendues.

Pour une preuve, on pourra se reporter à l'article \textit{Seven proofs of the
  Pearson Chi-squared independence test and its graphical interpretation} par
É. Benhamou et V. Melot (2018),
\href{https://arxiv.org/abs/1808.09171}{https://arxiv.org/abs/1808.09171}.  }
la proposition suivante : Soient deux variables aléatoires réelles $X$ et $Y$
indépendantes, ayant respectivement chacune $K$ et $L$ modes. Soit $n$ la
taille d'un échantillon aléatoire de $(X, Y)$ et $d_{\chi^2}$ la distance du
chi2 de la table de contingence de cet échantillon. Alors quand
$n \rightarrow +\infty,$
\[
  d_{\chi^2} \cvloi Z_{(K-1)(L-1)}.
\]


\begin{enumerate}[resume]
\item \textbf{Test du chi2}
  \begin{enumerate}
  \item Proposer un test statistique (hypothèses, statistique de test, région
    critique) permettant de tester l'hypothèse selon laquelle le traitement est
    efficace.
  \item Que peut-on dire de notre traitement sous $\alpha = 10\%$ ? $\alpha = 1\%$ ?
  \item \textbf{Fraude scientifique.} À un niveau de
    signification de $5\%$, à combien de personnes traitées faudrait-il trouver
    une bonne raison pour les exclure de l'étude afin de pouvoir rejeter l'hypothèse
    nulle et affirmer le succès du test ?
  \end{enumerate}
\end{enumerate}

Ce test s'appelle le test d'indépendance du chi2, et est implémenté dans \texttt{scipy.stats} :

  \begin{lstlisting}[language=Python]
    import scipy.stats as st
    st.chi2_contingency(np.array([[a00, a01], [a10, a11]]), correction=False)
  \end{lstlisting}
  
\begin{enumerate}[resume]
\item \textbf{Loi du chi2} \\
  \begin{enumerate}
  \item Quelle sont l'espérance et la variance de $Z_k$ ? 
  \item Soit $n \in \NN^*$, $0 < 1 < p_0$, et $N_0$ une variable aléatoire qui
    suit une loi binômiale de paramètres $n$ et $p_0$ : $N$ est la somme de $n$
    variables aléatoires réelles iid dont la loi est une loi de Bernoulli de
    paramètre $p_0$, et modélise le nombre de succès parmi $n$ tirages d'une
    telle variable de Bernoulli. Posons $N_1 = n - N_0$ et $p_1 = 1 -
    p_0$. Montrer que quand $n \rightarrow +\infty,$ 
    \[
      \frac{(N_0 - n p_0)^2}{n p_0}  + \frac{(N_1 - n p_1)^2}{n p_1} \cvloi Z_1.
    \]
  \end{enumerate}
\end{enumerate}

\begin{table}[h]
  \centering
  \begin{tabular}{|r|r|r|r|r|r|r|r|r|r|r|r|r|} \hline
      & .995 & .990 & .975 & .950 & .900 & .100 & .050  & .025  & .010 & 0.005 & 0.002 & 0.001 \\ \hline
    1 & 0.00 & 0.00 & 0.00 & 0.00 & 0.02 & 2.71 & 3.84  & 5.02  & 6.63 & 7.88 & 9.55 & 10.83 \\ \hline
    2 & 0.01 & 0.02 & 0.05 & 0.10 & 0.21 & 4.61 & 5.99  & 7.38  & 9.21 & 10.60 & 12.43 & 13.82 \\ \hline
    3 & 0.07 & 0.11 & 0.22 & 0.35 & 0.58 & 6.25 & 7.81  & 9.35  &  11.34 & 12.84 & 14.80 & 16.27 \\ \hline
    4 & 0.21 & 0.30 & 0.48 & 0.71 & 1.06 & 7.78 & 9.49  & 11.14 &  13.28 & 14.86 & 16.92 & 18.47 \\ \hline
    5 & 0.41 & 0.55 & 0.83 & 1.15 & 1.61 & 9.24 & 11.07 & 12.83 &  15.09 & 16.75 & 18.91 & 20.52 \\ \hline
  \end{tabular}
  \caption{Table du $\chi^2$ : Valeur de $z$ telle que $\PP(Z_k > z) = \alpha$ pour 
    plusieurs valeurs de $\alpha$ et pour $Z_k \sim \chi^2_k$.}
  \label{tab:chi2}
\end{table}


\end{document}

%%% Local Variables:
%%% mode: latex
%%% TeX-master: t
%%% End:
