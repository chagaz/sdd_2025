%-*- coding: utf-8 -*-
\label{chap:intro}

\paragraph{Notions :} statistique descriptive, statistique inférentielle,
apprentissage statistique, population

\paragraph{Objectifs pédagogiques :}
\begin{itemize}
\item Donner une définition de la science des données
\item Donner une définition de la statistique 
\item Donner une définition de l'apprentissage statistique, ou apprentissage
  automatique, ou encore \textit{machine learning}
\end{itemize}



\section{Qu'est-ce que la science des données ?}
La science des données, ou \textit{data science}, est un domaine dont la
définition dépend des personnes qui la donnent. On s'accorde néanmoins
généralement sur l'idée qu'il s'agit d'une science interdisciplinaire, qui
s'appuie sur les mathématiques (et notamment les probabilités, la statistique
et l'optimisation) et l'informatique (et notamment l'algorithmique, les bases
de données, l'architecture distribuée, et l'analyse numérique) mais aussi sur
des connaissances spécifiques au domaine d'application, autrement dit à la
nature des données étudiées (finance, commerce, physique, biologie, sociologie,
etc.).

C'est un domaine multiforme qui fait beaucoup parler de lui, et on se réfère
souvent à un article du \textit{Harvard Business Review} intitulé «
\textit{Data Scientist: the Sexiest Job of the 21st Century}
»\footnote{\href{https://hbr.org/2012/10/data-scientist-the-sexiest-job-of-the-21st-century}{https://hbr.org/2012/10/data-scientist-the-sexiest-job-of-the-21st-century}}.

La science des données permet par exemple de mieux comprendre les besoins de la
clientèle d'une entreprise ; de dimensionner des serveurs ; d'améliorer la
distribution de l'électricité ; d'analyser des données génomiques pour suggérer
de nouvelles hypothèses biologiques ; d'optimiser la livraison de colis ; de
détecter des fraudes ; de recommander des livres, films, ou autres produits
adaptés à nos goûts ; ou de personnaliser des publicités.

Dans les années à venir, la science des données, et en particulier le
\textit{machine learning}, nous permettra vraisemblablement d'améliorer la
sécurité routière (y compris grâce aux véhicules autonomes), la réponse
d'urgence aux catastrophes naturelles, le développement de nouveaux
médicaments, ou l'efficacité énergétique de nos bâtiments et industries.



\section{Objectifs de ce cours}
Ce cours se concentre sur les aspects mathématiques (statistique et
modélisation) et informatiques (utilisation pratique) de la science des
données. Il fait appel à des notions que vous avez découvertes en Probabilités,
en Optimisation, et en Outils Numériques pour les Mathématiques.

Le premier but de ce cours est de démystifier la science des données, le Big
Data, l'intelligence artificielle telle qu'on en parle de nos jours et de vous
donner les clés nécessaires à recevoir les informations sur le sujet d'un
\oe{}il critique.

Le deuxième but de ce cours est de poser les bases mathématiques et
algorithmiques de l'exploitation de données. Les domaines de la statistique
inférentielle et de l'apprentissage automatique sont vastes, et vous aurez, si
vous le souhaitez, amplement l'occasion de les explorer en deuxième et troisième
année.

\paragraph{Socle minimal} 
À l'issue de ce cours, vous devriez avoir acquis
\textit{a minima} les compétences suivantes :
\begin{enumerate}
\item Interpréter une p-valeur ;
\item Éviter les principaux écueils de visualisation de données ;
\item Expliciter le principe de la minimisation du risque empirique, et
  l'illustrer sur un ou deux exemples d'algorithmes d'apprentissage automatique ;
\item Expliquer comment un algorithme d'apprentissage automatique peut
  reproduire un biais  ;
\item Reconnaître une situation pouvant prêter au surapprentissage ;
\item Sélectionner et valider un modèle d'apprentissage automatique ;
\item Décrire un réseau de neurones artificiels comme un modèle paramétrique ;
\item Lister quelques succès et limites de l'apprentissage profond ;
\item Donner des exemples d'algorithmes d'apprentissage automatique permettant
  d'apprendre des modèles non linéaires ;
\item Discuter de quelques uns des enjeux éthiques liés à l'intelligence
  artificielle.
\end{enumerate}

Les sections marquées d'un $\bullet$ dans le poly sont celles qui ne font pas
partie de ce socle minimal. Ne pensez pas cependant qu'elles soient plus
difficiles ! Les sections marquées de $\bullet \bullet$ permettent d'aller plus
loin et sont considérées hors programme.

\paragraph{Acquisition de données}
Aucune approche statistique ne pourra créer un bon modèle à partir de données
qui ne sont pas pertinentes -- c'est le concept {\it garbage in, garbage out}
qui stipule qu'un algorithme d'apprentissage auquel on fournit des données de
mauvaise qualité ne pourra rien en faire d'autre que des prédictions de
mauvaise qualité.

Bien que ce cours soit consacré aux outils mathématiques et, dans une moindre
mesure, informatiques de la science des données, il ne faut pas négliger qu'une
part importante du travail de {\it machine learner} ou de {\it data scientist} est un
travail d'ingénierie consistant à préparer les données afin d'éliminer les
données aberrantes, gérer les données manquantes, choisir une représentation
pertinente, etc.

\paragraph{Big data} De plus en plus, les quantités de données disponibles
imposent de transformer les algorithmes utilisés et de faire appel à des
architectures de calcul et de base de données distribuées. Nous n'abordons pas
non plus ce point dans ce cours. Le cours optionnel Large Scale Machine
Learning, proposé en semaine bloquée au printemps, vous permettra de découvrir
ce domaine.

\section{Qu'est-ce que la statistique ?}

Le terme de « statistique » est dérivé du latin « \textit{status} » (signifiant
« état »). Historiquement, \textbf{les statistiques} concernent l'étude
méthodique, par des procédés numériques (inventaires, recensements, etc.) des
faits sociaux qui définissent un état. % Elles sont désormais utilisées dans tous
% les secteurs où l'on dispose de données : sciences sociales mais aussi santé,
% environnement, industrie, économie, recherche scientifique, etc.

Par contraste, \textbf{la statistique} est un ensemble de méthodes des
mathématiques appliquées permettant de décrire et d'analyser des phénomènes
dont la nature rend une étude exhaustive de tous leurs facteurs impossible. Ces
méthodes permettent d'étudier des données, ou observations, consistant en la
mesure d'une ou plusieurs caractéristiques d'un ensemble de personnes ou objets
équivalents.

\subsection{Vocabulaire}

L'ensemble de personnes ou d'objets équivalents étudiés est appelé \textbf{la
  population.} Il peut s'agir d'une population au sens « courant » du terme
(par exemple, l'ensemble de la population française, ou l'ensemble des
individus d'une espèce animale sur un territoire) mais aussi plus largement
d'un ensemble plus générique d'objets que l'on cherche à étudier (par exemple,
l'ensemble des pièces produites par une chaîne de montage, un ensemble de
particules en physique, etc.)

Chacun des éléments de la population est appelé \textbf{individu.} 

Les caractéristiques que l'on mesure pour chacun de ces individus sont appelées
les \textbf{variables} ; les individus pour lesquels ces caractéristiques ont
été mesurées sont appelées des \textbf{observations}. Un ensemble de $n$
observations $(x_1, x_2, \dots, x_n)$ d'une variable est appelée \textbf{série
  statistique.}

Par exemple, si j'étudie les données climatiques pour la station météo de
Paris-Montsouris en 2019 (cf. tableau~\ref{tab:meteo_data}), il s'agit d'une
population de 365 individus. Cette population peut contenir 8 variables :
températures minimale, maximale et moyenne ; vitesse maximale du vent ;
ensoleillement ; précipitations totales ; pressions atmosphériques minimale et
maximale.


Lorsque la population à étudier est trop grande pour qu'il soit possible
d'observer chacun de ses individus, on étudie alors une partie seulement de la
population. Cette partie est appelée \textbf{échantillon}. On parle alors de
\textbf{sondage}, par opposition à un \textbf{recensement}, qui consiste à
étudier tous les individus d'une population. Nous reviendrons sur la notion
d'échantillon dans le chapitre~\ref{chap:estimation}.

Par exemple, la population des élèves de première année des Mines est
composée de 128 individus. Si je recueille l'âge, le département de naissance
et le nombre de frères et s\oe{}urs de 20 de ces élèves, j'aurai mesuré 3 variables
sur un échantillon de 20 observations. 

On distinguera plusieurs types de variables :
\begin{itemize}
\item les \textbf{variables quantitatives} : des caractéristiques numériques
  qui s'expriment naturellement à l'aide de nombres réels. Ces variables
  peuvent être \textbf{discrètes} si le nombre de valeurs qu'elles peuvent
  prendre est fini ou dénombrable (ex : âge, nombre de frères et s\oe{}urs) ou
  \textbf{continues} (ex : températures, taille, pression atmosphérique)
\item les \textbf{variables qualitatives} : des caractéristiques qui, bien
  qu'elles puissent être encodées numériquement (ex : département de
  naissance), relèvent plutôt de catégories et sur lesquelles les opérations
  arithmétiques de base (somme, moyenne) n'ont aucun sens. On parle de
  variables \textbf{nominales} s'il n'y a pas d'ordre total sur l'ensemble de
  ces catégories (ex : département de naissance) ou \textbf{ordinales} s'il y
  en a (ex : entièrement d'accord, assez d'accord, pas vraiment d'accord, pas
  du tout d'accord).
\end{itemize}

Remarque : seuiller des variables quantitatives permet de les transformer en
variables qualitatives ordinales. Par exemple, une variable d'âge peut être
transformée en catégories (< 18, 18 -- 20, 20 -- 35, etc.)

Le tableau~\ref{tab:remboursement_data} montre un exemple d'un échantillon de
20 individus d'une population de données de remboursements d'un acte biologique
bien précis : le dosage de l'antigène tumoral 125. Ces données sont issues de
la base de dépenses de biologie médicale en France mise à disposition par
l'Assurance
Maladie\footnote{\url{http://open-data-assurance-maladie.ameli.fr/biologie/index.php}}. La
population complète, de 604 individus, est disponible dans le fichier\\
\texttt{notebooks/data/OPEN\_BIO\_2018\_7325.csv}.

Chaque individu de cette population (i.e. ligne du tableau) correspond à un
ensemble de dosages et est décrit par 5 variables : la tranche d'âge des
patients et patientes ; leur région ; le nombre de dosages ; et enfin les
montants remboursés et remboursables. Dans ce tableau, l'âge est une variable
qualitative ordinale ; la région une variable qualitative ; et les nombres et
montants des variables quantitatives. On pourra choisir de traiter le nombre de
remboursements comme une variable discrète ou continue.


\subsection{Statistique descriptive} 
La \textbf{statistique descriptive,} aussi appelée \textbf{statistique
  exploratoire,} consiste à caractériser une population par la détermination
d'un certain nombre de grandeurs qui la décrivent. Son objectif est de
synthétiser l'information contenue dans un ensemble d'observations et de mettre
en évidence des propriétés de cet ensemble. Elle permet aussi de suggérer des
hypothèses relatives à la population dont sont issues les observations. Il
s'agit principalement de calculer des indicateurs (par exemple des moyennes) et
de visualiser les données par des graphiques. La visualisation peut être
enrichie par des techniques d'apprentissage non-supervisé (cf
section~\ref{sec:nonsup_ml}) qui permettent de réduire le nombre de variables
ou de regrouper ensemble les individus semblables.  La statistique descriptive
est traitée au chapitre~\ref{chap:stat_descr}.

\subsection{Statistique inférentielle} 
Aussi appelée \textbf{statistique décisionnaire,} ou encore \textbf{inférence
  statistique,} la \textbf{statistique inférentielle} consiste à tirer des
conclusions sur une population à partir de l'étude d'un échantillon de
celle-ci. Les données observées sont considérées comme un échantillon d'une
population. Il s'agit alors d'étendre des propriétés constatées sur
l'échantillon à la population. L'inférence statistique repose beaucoup sur les
probabilités : on considèrera les observations comme les réalisations de
variables aléatoires, ce qui permettra d'approcher les caractéristiques
probabilistes de ces variables aléatoires à l'aide d'indicateurs calculés sur
l'échantillon. Le chapitre~\ref{chap:estimation} traite de statistique
inférentielle.

\section{Qu'est-ce que l'apprentissage statistique ?}
Qu'est-ce qu'apprendre, comment apprend-on, et que cela signifie-t-il pour une
machine ? La question de l'{\it apprentissage} fascine les spécialistes de
l'informatique et des mathématiques tout autant que neurologues, pédagogues,
philosophes ou artistes.

Dans le cas d'un programme informatique, on parle d'\textbf{apprentissage
  statistique,} ou \textbf{apprentissage automatique}, ou encore {\it machine
  learning}, quand ce programme a la capacité de se modifier lui-même sans que cette
modification ne soit explicitement programmée. Cette définition est celle
donnée par Arthur Samuel en 1959. On peut ainsi opposer un programme {\it
  classique}, qui utilise une procédure et les données qu'il reçoit en entrée
pour produire en sortie des réponses, à un programme {\it d'apprentissage
  automatique}, qui utilise les données et les réponses afin de produire la
procédure qui permet d'obtenir les secondes à partir des premières.

Supposons par exemple qu'une entreprise veuille connaître le montant total
dépensé par un client ou une cliente à partir de ses factures. Il suffit
d'appliquer un algorithme classique, à savoir une simple addition : un
algorithme d'apprentissage n'est pas nécessaire.

Par contraste, supposons maintenant que l'on veuille utiliser ces factures pour
déterminer quels produits le client est le plus susceptible d'acheter dans un
mois. Bien que cela soit vraisemblablement lié, nous n'avons manifestement pas
toutes les informations nécessaires pour ce faire. Cependant, si nous disposons
de l'historique d'achat d'un grand nombre d'individus, il devient possible
d'utiliser un algorithme d'apprentissage automatique pour qu'il en tire un
modèle prédictif nous permettant d'apporter une réponse à notre question.

Ce point de vue informatique sur l'apprentissage automatique justifie que l'on
considère qu'il s'agit d'un domaine différent de celui de la
statistique. Cependant, nous aurons l'occasion de voir que la frontière entre
inférence statistique et apprentissage est souvent mince.  Il s'agit ici,
fondamentalement, de \textbf{modéliser} un phénomène à partir de données
considérées comme autant d'observations de celui-ci. \pagebreak
\begin{attention}
  Bien que l'usage soit souvent d'appeler les deux du même nom, il faut
  distinguer l'\textbf{algorithme d'apprentissage} automatique du
  \textbf{modèle appris} : le premier utilise les données pour produire le
  second, qui peut ensuite être appliqué comme un programme classique.
\end{attention}

On distingue plusieurs types de problèmes en apprentissage automatique. Nous nous intéresserons dans ce cours à l'apprentissage supervisé et à l'apprentissage non-supervisé, en ignorant entre autres l'apprentissage par renforcement principalement utilisé en robotique.
\subsection{Apprentissage supervisé}
L'\textbf{apprentissage supervisé,} ou \textbf{apprentissage prédictif,} est peut-être le type de problèmes de machine
learning le plus facile à appréhender : son but est d'apprendre à faire des
{\it prédictions}, à partir d'une liste d'exemples \textbf{étiquetés},
c'est-à-dire accompagnés de la valeur à prédire. Les étiquettes servent de \og
prof \fg~et supervisent l'apprentissage de l'algorithme. Un exemple classique est celui de l'annotation d'images : il s'agit par exemple de déterminer si une image représente ou non un chat.

Étant données $n$ {observations} $x_1, x_2, \ldots, x_n$ appartenant à un
espace $\XX$, et leurs {étiquettes} $y_1,$ $y_2, \ldots, y_n$ appartenant à un
espace $\YY$, on suppose que les étiquettes peuvent être obtenues à partir des
observations grâce à une fonction $\phi \colon \XX \to \YY$ fixe et inconnue
: $y_i = \phi(x_i) + \epsilon_i$, où $\epsilon_i$ est un bruit.  Il
s'agit alors d'utiliser les données pour déterminer une fonction
$f \colon \XX \to \YY$ telle que, pour tout couple
$(x, \phi(x)) \in \XX \times \YY$, $f(x) \approx \phi(x)$. On suppose généralement pour cela que les couples $(x_i, y_i)$ sont les réalisations d'un vecteur aléatoire $(X, Y)$ vérifiant $Y = \phi(X) + \epsilon$ et l'on cherche à déterminer $\phi$.

Nous aborderons en détail l'apprentissage supervisé dans les chapitres~\ref{chap:erm} à~\ref{chap:nonlin}.

\subsection{Apprentissage non supervisé}
\label{sec:nonsup_ml}
Dans le cadre de l'\textbf{apprentissage non supervisé}, les données ne sont
pas étiquetées. Il s'agit alors de modéliser les observations pour mieux les
comprendre. Ces techniques sont ainsi complémentaires de celles de la
statistique exploratoire.

Parmi les exemples d'apprentissage non supervisé, on compte notamment 
\begin{itemize}
\item la \textbf{réduction de dimension}, que nous aborderons au
  chapitre~\ref{chap:dimred}, qui permet de créer un petit nombre de variables
  qui « résument » les mesures prises sur les observations. Il s'agit en fait de trouver
  une représentation des données dans un espace de dimension plus faible que
  celle de l'espace dans lequel elles sont représentées initialement. Cela
  permet de réduire les temps de calcul et l'espace mémoire nécessaire au
  stockage les données, mais aussi souvent d'améliorer les performances d'un
  algorithme d'apprentissage supervisé entraîné par la suite sur ces données.
\item le \textbf{partitionnement}, ou \textbf{clustering}, qui permet de
  réduire la taille d'un échantillon en regroupant les individus présentant des
  caractéristiques homogènes. Nous ne traiterons pas de ce sujet dans ce
  cours. Il sera notamment abordé dans le cours optionnel d'apprentissage
  automatique proposé en semaine bloquée à l'automne (S1333-5).
\end{itemize}


\section{Et l'intelligence artificielle, dans tout ça ?}
Le machine learning est une branche de l'intelligence
artificielle. En effet, un système incapable d'apprendre peut difficilement
être considéré comme intelligent. 
L'intelligence artificielle, que l'on peut définir comme l'ensemble des techniques mises en
{\oe}uvre afin de construire des machines capables de faire preuve d'un
comportement que l'on peut qualifier d'intelligent, fait aussi appel aux
sciences cognitives, à la neurobiologie, à la logique, à l'électronique, à
l'ingénierie et bien plus encore.

Le terme d'\og intelligence artificielle \fg~stimulant plus
l'imagination, il est de plus en plus souvent employé en lieu et
place de celui d'apprentissage automatique.

\section{Bonnes pratiques}
L'essor récent de l'intelligence artificielle, à travers notamment les
développements en apprentissage profond que nous aborderons brièvement au
chapitre~\ref{chap:nonlin}, suscite de vifs débats philosophiques, éthiques et
moraux dans notre société. Sans entrer en profondeur dans ces débats, ce qui
relèverait d'un cours d'éthique ou de philosophie et non plus d'un cours de
mathématiques appliquées, nous en aborderons quelques points clés dans le
chapitre~\ref{chap:pratiques}, dédiée aux bonnes pratiques en science des
données. Il serait malhonnête dans ce cours de prétendre pouvoir
détacher les mathématiques et l'informatique du contexte de leur utilisation.


\section{Sources}
Le contenu de ce poly s'appuie en partie sur des documents mis à disposition en
ligne par Stéphane Canu, Laure Reboul, et Joseph Salmon, que je remercie
vivement, ainsi que les ouvrages \textit{Probabilités, analyse des données et
  Statistique} (Technip) de Gilbert Saporta et mon \textit{Introduction au
  Machine Learning} (Dunod InfoSup). Vous trouverez une version PDF (sans les
exercices) de ce dernier sur ma page web \href{http://cazencott.info}{http://cazencott.info}.


\begin{plusloin}
  \item L'article \href{https://www.tandfonline.com/doi/full/10.1080/10618600.2017.1384734}{\textit{50 Years of Data Science} de David Donaho}
aborde les différences entre statistique, science des données, et apprentissage
automatique et donne une vision d'ensemble de ces domaines.
\end{plusloin}

%\section{Tableaux de données}
\begin{table}
  \centering
  \begin{tabular}[h]{|c|c|c|c|c|c|c|c|} \hline
    T min & T max & T moy & Vent & Ensoleillement & Précipitations & P min & P max \\
    \textdegree C & \textdegree C & \textdegree C & km/h & min & mm & hPa & hPa \\ \hline
    7.6 & 9.6 & 8.4 & 22.2 & 0 & 0 & 1034 & 1036.6 \\ \hline
    5.6 & 7.2 & 6.3 & 24.1 & 0 & 0 & 1037.3 & 1041.3 \\ \hline
    4.1 & 6.6 & 5.4 & 16.7 & 0 & 0 & 1040.2 & 1041.8 \\ \hline
    3.1 & 6 & 4.7 & 20.4 & 0 & 0 & 1039.5 & 1041.7 \\ \hline
    4.2 & 5.9 & 5 & 20.4 & 0 & 0 & 1037.5 & 1039.6 \\ \hline
    4.3 & 6.8 & 5.6 & 16.7 & 0 & 0 & 1036.5 & 1038 \\ \hline
    6.8 & 8.6 & 7.4 & 20.4 & 0 & 0.6 & 1030.5 & 1037.2\\ \hline
    7.4 & 9.7 & 8.5 & 24.1 & 120 & 0 & 1025.9 & 1029.7 \\ \hline
    4 & 7.7 & 5.2 & 29.6 & 42 & 0.8 & 1024.1 & 1026.4 \\ \hline 
    2.1 & 5.5 & 4 & 18.5 & 30 & 0 & 1026.6 & 1029.5 \\ \hline
    4.2 & 8.3 & 6.2 & 14.8 & 0 & 1.2 & 1028.1 & 1030.5 \\ \hline
    6.7 & 9.1 & 8 & 22.2 & 0 & 0.8 & 1021.7 & 1030.6 \\ \hline
    8.8 & 11.9 & 10.5 & 31.5 & 30 & 0.8 & 1014 & 1021.1 \\ \hline
    8.5 & 10.9 & 8.8 & 29.6 & 0 & 0 & 1014 & 1024.8 \\ \hline
    6.9 & 8.6 & 7.6 & 16.7 & 0 & 0 & 1020.3 & 1025.3 \\ \hline
    1.9 & 7.8 & 5.2 & 27.8 & 276 & 3 & 1007.9 & 1019.6 \\ \hline
    4 & 8.5 & 5.4 & 27.8 & 0 & 0 & 1007.5 & 1019.7 \\ \hline
    0.9 & 6.1 & 2.4 & 18.5 & 342 & 0 & 1017.2 & 1021 \\ \hline
    -1.7 & 2.8 & 0.3 & 14.8 & 78 & 4 & 1009.7 & 1016.8 \\ \hline
    1.9 & 3 & 2.5 & 20.4 & 0 & 0.8 & 1010.1 & 1021.8 \\ \hline
    -2.2 & 3.6 & 0.1 & 13 & 480 & 0 & 1019.2 & 1024.9 \\ \hline
    -2.4 & 1.7 & -0.1 & 20.4 & 0 & 7.4 & 998.4 & 1018.3 \\ \hline
    0.6 & 2.1 & 1.3 & 24.1 & 6 & 1 & 995.6 & 1007.4 \\ \hline
    -0.4 & 2.5 & 1.2 & 20.4 & 0 & 1 & 1008.4 & 1015.5 \\ \hline
    1.7 & 5.5 & 3.9 & 13 & 0 & 1.2 & 1015.2 & 1018.4 \\ \hline
    5.5 & 9.6 & 8.4 & 29.6 & 24 & 1.6 & 998.1 & 1016.9 \\ \hline
    4.4 & 8.2 & 6.4 & 37 & 6 & 4.4 & 989.8 & 1001.4 \\ \hline
    2.7 & 6.9 & 4.4 & 25.9 & 216 & 0 & 1001.7 & 1011.7 \\ \hline
    -0.8 & 5.2 & 1.7 & 24.1 & 18 & 19.6 & 989.5 & 1011.7 \\ \hline
    0.5 & 5.2 & 2.3 & 33.3 & 252 & 0.4 & 990 & 998.9 \\ \hline 
    -1 & 2.5 & 1.3 & 25.9 & 24 & 1.2 & 983.5 & 998 \\ \hline
  \end{tabular}
  \caption{Exemple de 8 variables pour 31 observations (celles du mois de janvier 2019) 
    de la population des données climatiques pour la station météo de Paris-Montsouris. Ces données sont disponibles dans le fichier \texttt{data/meteo\_data.csv}.}
  \label{tab:meteo_data}
\end{table}

\begin{table}
  \centering
  \begin{tabular}{|r|r|r|r|r|r|} \hline
    \multirow{2}{*}{Âge} & \multirow{2}{*}{Région} & Nombre & Montants & Montants  \\ 
    & & d'actes & remboursés & remboursables \\ 
    ans & & & (\texteuro) & (\texteuro) \\\hline 
    $>$ 60 & 76 & 26 & 377,96 & 402,80\\  \hline
    $>$ 60 & 75 & 1\,401 & 14\,054,37 & 21\,332,15\\  \hline
    $>$ 60 & 44 & 5\,299 & 65\,928,93 & 80\,447,00\\  \hline
    $>$ 60 & 32 & 1\,706 & 25\,137,65 & 26\,032,65\\  \hline
    $>$ 60 & 32 & 2\,596 & 37\,877,02 & 39\,336,15\\  \hline
    $>$ 60 & 27 & 14 & 159,85 & 211,35\\  \hline
    $>$ 60 & 24 & 3\,565 & 50\,770,46 & 54\,076,15\\  \hline
    $>$ 60 & 11 & 396 & 5\,226,55 & 6\,060,05\\  \hline
    $>$ 60 & 5 & 260 & 4\,496,91 & 4\,676,40\\  \hline
    $>$ 60 & 93 & 162 & 2\,303,56 & 2\,466,10\\  \hline
    $>$ 60 & 76 & 578 & 8\,499,53 & 8\,793,10\\  \hline
    40-59 & 76 & 13 & 172,26 & 199,80\\  \hline
    40-59 & 44 & 102 & 1\,204,93 & 1\,557,20\\  \hline
    40-59 & 11 & 48 & 555,39 & 733,05\\  \hline
    40-59 & 84 & 14 & 190,21 & 217,85\\  \hline
    40-59 & 32 & 126 & 1\,350,06 & 1\,912,15\\  \hline
    20--39 & 32 & 749 & 7\,941,69 & 11\,362,40\\  \hline
    20--39 & 32 & 24 & 289,35 & 365,25\\  \hline
    20--39 & 5 & 918 & 9\,704,10 & 16\,550,40\\  \hline
    20--39 & 11 & 106 & 1\,073,32 & 1\,618,35\\  \hline
  \end{tabular}
  \caption{Population de remboursements du dosage de l'antigène 125 dans le sang en 2018, 
    composée de 20 individus décrits par 5 variables et extraite du fichier \texttt{data/OPEN\_BIO\_2018\_7325.csv}. \\
    Région : 5 = Régions et Départements d'outre-mer. 
    11 = Ile-de-France. 
    24 = Centre-Val de Loire.
    27 = Bourgogne-Franche-Comté.
    32 = Hauts-de-France.
    44 = Grand-Est.
    75 = Nouvelle-Aquitaine.
    76 = Occitanie.
    84 = Auvergne-Rhône-Alpes.
    93 = Provence-Alpes-Côte d'Azur et Corse.}
  \label{tab:remboursement_data}
\end{table}



%%% Local Variables:
%%% mode: latex
%%% TeX-master: "../sdd_2025_poly"
%%% End: