%-*- coding: utf-8 -*-
\section{QCM}
\paragraph{Question 1.} Un modèle de régression régularisée est plus
  susceptible de surapprendre si le paramètre de régularisation est
\begin{itemize}
\item[$\square$] élevé ;
\item[$\square$] faible ;
\item[$\square$] ça dépend des cas.
\end{itemize}

\paragraph{Question 2.}     Dans un lasso, il y a plus de coefficients nuls quand le 
    paramètre de régularisation est 
\begin{itemize}
\item[$\square$] élevé ;
\item[$\square$] faible ;
\item[$\square$] ça dépend des cas.
\end{itemize}

\paragraph{Question 3.} Par rapport à un modèle complexe, un modèle plus simple est
\begin{itemize}
\item[$\square$] plus rapide à entraîner ;
\item[$\square$] plus susceptible de surapprendre ;
\item[$\square$] plus susceptible de bien généraliser ;
\item[$\square$] plus susceptible de minimiser le risque empirique.
\end{itemize}


\section*{Solution}
{%
\noindent
\rotatebox[origin=c]{180}{%
\noindent
\begin{minipage}[t]{\linewidth}
\paragraph{Question 1.}  Quand $\lambda$ est faible, c'est le risque empirique
qui domine et le modèle est plus susceptible de surapprendre. \newline

\paragraph{Question 2.} Quand $\lambda$ croît, le régulariseur prend plus
d'importance et le nombre de coefficients nuls augmente. \newline

\paragraph{Question 3.} Le temps d'entrainement ne dépend pas toujours de la
complexité du modèle. Un modèle plus simple sera cependant souvent plus
rapide à entrainer.

Un modèle simple est moins susceptible de surapprendre (et plus susceptible de sousapprendre) ; généralisera mieux, sauf s'il est en sous-apprentissage ; et minimisera moins bien le risque empirique.
\end{minipage}%
}%


%%% Local Variables:
%%% mode: latex
%%% TeX-master: "../../sdd_2025_poly"
%%% End:
