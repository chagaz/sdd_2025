%-*- coding: utf-8 -*-
\section{QCM}
\paragraph{Question 1.} Quand le nombre d'observations tend vers l'infini, 
\begin{itemize}
\item[$\square$] le risque empirique d'un modèle converge vers le risque de ce modèle ; 
\item[$\square$] le risque empirique minimal converge vers le risque minimal ; 
\item[$\square$] le minimiseur du risque empirique converge vers le minimiseur du risque.
\end{itemize}

\paragraph{Question 2.} Supposons un problème de classification en 2
dimensions, avec $n$ observations. Nous considérons comme espace des hypothèses
l'ensemble des unions de $K$ cercles ($K > 0$ est fixé) : les points intérieurs
à ces cercles sont étiquetés positifs, les autres négatifs. Alors
\begin{itemize}
\item[$\square$] Il ne s'agit pas d'un modèle paramétrique.
\item[$\square$] Il s'agit d'un modèle paramétrique à $K$ paramètres.
\item[$\square$] Il s'agit d'un modèle paramétrique à $2K$ paramètres.
\item[$\square$] Il s'agit d'un modèle paramétrique à $3K$ paramètres.
\end{itemize}

\paragraph{Question 3.} Quel algorithme préférer pour entraîner une régression
linéaire sur un jeu de données contenant $n$ observations et $p$ variables :
\begin{itemize}
\item Si $n=10^5$ et $p=5$ ?
  \begin{itemize}
  \item[$\square$] Une inversion de matrice.
  \item[$\square$] Un algorithme du gradient.
  \end{itemize}
\item Si $n=10^5$ et $p=10^5$ ?
  \begin{itemize}
  \item[$\square$] Une inversion de matrice.
  \item[$\square$] Un algorithme du gradient.
  \end{itemize}
\end{itemize}

\section*{Solution}
{%
\noindent
\rotatebox[origin=c]{180}{%
\noindent
\begin{minipage}[t]{\linewidth}
\paragraph{Question 1.} Seule la première proposition est vraie. \newline

\paragraph{Question 2.} Il s'agit d'un modèle paramétrique et nous avons besoin
de $3K$ paramètres pour déterminer les coordonnées de $K$ cercles (coordonnées
du centre + rayon). \newline

\paragraph{Question 3.}  Lorsque la matrice $X^\top X$ (de dimensions
$p \times p$) est de petite taille (peu de variables), on pourra utiliser un
algorithme d'inversion de matrice. Sinon, un algorithme du gradient sera plus
approprié.
\end{minipage}%
}%

%%% Local Variables:
%%% mode: latex
%%% TeX-master: "../../sdd_2025_poly"
%%% End:
