%-*- coding: utf-8 -*-
\section{QCM}

\paragraph{Question 1.} Dans une étude dont le but est de déterminer si la
consommation de chocolat noir améliore les résultats scolaires, l'hypothèse
alternative est :
\begin{itemize}
\item[$\square$] Les élèves qui consomment du chocolat noir obtiennent les
  mêmes résultats que les élèves qui n'en consomment pas.
\item[$\square$] Les élèves qui consomment du chocolat noir obtiennent de moins
  bons résultats que les élèves qui n'en consomment pas.
\item[$\square$] Les élèves qui consomment du chocolat noir obtiennent de
  meilleurs résultats que les élèves qui n'en consomment pas.
\item[$\square$] Les élèves qui consomment du chocolat noir obtiennent des
  résultats différents des élèves qui n'en consomment pas.
\end{itemize}

\paragraph{Question 2.} La p-valeur est :
\begin{itemize}
\item[$\square$] La probabilité que l'hypothèse nulle soit vraie.
\item[$\square$] La probabilité que l'hypothèse alternative soit vraie.
\item[$\square$] La probabilité, si l'hypothèse nulle est vraie, d'obtenir une
  situation au moins aussi surprenante que celle observée.
\item[$\square$] La probabilité, si l'hypothèse alternative est vraie,
  d'obtenir une situation au moins aussi surprenante que celle observée.
\end{itemize}

\section*{Solution}
{%
\noindent
\rotatebox[origin=c]{180}{%
\noindent
\begin{minipage}[t]{\linewidth}
\paragraph{Question 1.}
L'hypothèse alternative est que les élèves qui consomment du chocolat noir
obtiennent de meilleurs résultats que les élèves qui n'en consomment pas. Il
s'agit d'un test unilatéral. Le test sera plus puissant que si on utilise
l'hypothèse alternative bilatérale (les résultats entre les deux groupes sont
différents). Cependant, si ce test unilatéral ne permet pas de rejeter
l'hypothèse nulle, il sera impossible de déterminer si c'est parce qu'il n'y a
pas de différence entre les deux groupes ou parce que l'effet est dans l'autre
sens.

Remarquons enfin que corrélation n'est pas causalité : ce test permet de
déterminer si la différence de performance entre les élèves cacaophiles et les
autres est significative, mais en aucun cas si elle est \textit{due} à la
consommation de chocolat. Il est tout à fait possible que la consommation de
chocolat noir soit liée à d'autres facteurs (en particulier sociaux) qui eux
influent sur la réussite scolaire. \newline

\paragraph{Question 2.} La p-valeur est la probabilité d'obtenir une situation
au moins aussi surprenante que celle observée si l'hypothèse nulle est
vraie. Il est très important de ne pas l'interpréter comme la probabilité que
l'hypothèse nulle soit vraie : 
\[
  \PP(t \geq t_0 | \HH_0) \neq \PP(\HH_0 | t \geq t_0). 
\]
\end{minipage}%
}%

%%% Local Variables:
%%% mode: latex
%%% TeX-master: "../../sdd_2025_poly"
%%% End:
